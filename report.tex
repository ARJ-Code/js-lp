\documentclass{article}
\usepackage[hidelinks]{hyperref}
\usepackage{graphicx}
\usepackage{titling}
\usepackage{float}
\usepackage[text={18cm,21cm},centering]{geometry}
\usepackage{hyperref} \hypersetup{ colorlinks=true, linkcolor=blue, filecolor=magenta,
urlcolor=blue, }

\begin{document}


\begin{titlepage}
    \centering
    {\bfseries\LARGE Universidad de La Habana \par}
    \vspace{1cm}
    {\scshape\Large Facultad de Matemática y Computación \par}
    \vspace{3cm}
    {\scshape\Huge Lenguajes de Programación\\ Seminario 4 (Javascript) \par}
    \vfill

    {\Large Juan Carlos Espinoza Delgado C-411 \par}
    {\Large Raudel Alejandro Gómez Molina C-411 \par}
    {\Large Alex Sierra Alcalá C-411 \par}
    \vfill
    {\href{https://github.com/ARJ-Code/js-lp}{Proyecto en github} \par}
\end{titlepage}

\section*{Introducción}

\subsection*{Propósito del Surgimiento de JavaScript}

JavaScript nació en 1995 de la mano de Brendan Eich en Netscape Communications, con el propósito de permitir a los
desarrolladores agregar interactividad a las páginas web. A diferencia de lenguajes como Java, que requerían
compilación y eran más complejos, JavaScript fue diseñado para ser sencillo, interpretado y embebido directamente en
los navegadores. Su objetivo principal era facilitar la manipulación del Document Object Model (DOM) y permitir la
creación de experiencias web más dinámicas y atractivas para los usuarios.

\subsection*{Problemas de Compatibilidad de los Navegadores Antes de la Adopción de ECMAScript 6}

Antes de la estandarización y adopción de ECMAScript 6 (ES6) en 2015, el ecosistema de JavaScript enfrentaba serios
desafíos de compatibilidad entre navegadores. Cada navegador (Internet Explorer, Netscape, Firefox, Chrome, Safari,
etc.) implementaba el lenguaje de manera ligeramente diferente, lo que resultaba en:

\begin{itemize}
    \item Inconsistencias en las API y Funcionalidades: Los desarrolladores tenían que escribir código específico
          para cada navegador o usar librerías como jQuery para abstraer estas diferencias.
    \item Problemas de Rendimiento: Diferentes motores de JavaScript (como V8 en Chrome, SpiderMonkey en Firefox)
          tenían variaciones significativas en cómo ejecutaban el código, lo que afectaba la experiencia del usuario.
    \item Falta de Características Modernas: Sin una actualización estandarizada, los desarrolladores no podían
          aprovechar características modernas y eficientes en todos los navegadores, limitando la capacidad de innovación y la complejidad de las aplicaciones web.

\end{itemize}

\subsection*{Actual Uso y Versatilidad de JavaScript}

Desde la adopción de ES6 y posteriores actualizaciones de ECMAScript, JavaScript ha evolucionado enormemente,
convirtiéndose en uno de los lenguajes de programación más versátiles y ampliamente utilizados en la industria del
software. Actualmente, JavaScript es la columna vertebral del desarrollo web moderno, gracias a sus siguientes
características:

\begin{itemize}
    \item Compatibilidad y Estandarización: Con la adopción de ES6 y versiones posteriores, los navegadores modernos
          ofrecen una implementación más consistente y completa del lenguaje, reduciendo significativamente los problemas
          de compatibilidad.
    \item Versatilidad: JavaScript no solo se utiliza en el desarrollo frontend con tecnologías como React, Angular y
          Vue.js, sino también en el backend con Node.js, lo que permite a los desarrolladores usar un solo lenguaje en
          toda la stack de aplicaciones.
    \item Ecosistema Rico: Un vasto ecosistema de herramientas, librerías y frameworks facilita el desarrollo rápido
          y eficiente de aplicaciones web, móviles (React Native), de escritorio (Electron) e incluso en Internet de las
          Cosas (IoT).
    \item  Innovación Continua: La comunidad de JavaScript y los comités de ECMAScript siguen introduciendo nuevas
          características y mejoras que mantienen al lenguaje relevante y poderoso para enfrentar los desafíos tecnológicos
          actuales.
\end{itemize}

En resumen, JavaScript ha recorrido un largo camino desde sus inicios como un simple lenguaje de scripting para
navegadores, hasta convertirse en una pieza fundamental del desarrollo de software moderno. Su evolución ha permitido
resolver problemas críticos de compatibilidad y ha potenciado su uso en múltiples contextos, demostrando una
versatilidad y adaptabilidad que siguen siendo esenciales en la industria tecnológica actual.

\section*{Problemas Estructurales de JavaScript y el Surgimiento de TypeScript}

\subsection*{Problemas Estructurales de JavaScript}

A pesar de su popularidad y versatilidad, JavaScript presenta varios problemas estructurales que pueden complicar
el desarrollo de aplicaciones complejas. Estos problemas incluyen:

\begin{enumerate}
    \item Tipado Dinámico y Falta de Tipos Estáticos:
          \begin{itemize}
              \item JavaScript es un lenguaje de tipado dinámico, lo que significa que las variables pueden cambiar de
                    tipo durante la ejecución. Esto puede llevar a errores difíciles de detectar y depurar.
              \item La falta de verificación de tipos en tiempo de compilación puede resultar en problemas de
                    consistencia y errores de tipo que solo se manifiestan en tiempo de ejecución.
          \end{itemize}
    \item Problemas de Mantenimiento y Escalabilidad:
          \begin{itemize}
              \item En proyectos grandes, la ausencia de un sistema de tipos robusto puede hacer que el código sea
                    difícil de mantener y refactorizar.
              \item La gestión de grandes bases de código puede volverse compleja debido a la falta de estructura y
                    organización que un sistema de tipos podría proporcionar.
          \end{itemize}
    \item Herencia Prototípica:
          \begin{itemize}
              \item Aunque JavaScript soporta herencia a través de prototipos, esta puede ser menos intuitiva y más
                    propensa a errores que la herencia basada en clases, especialmente para desarrolladores
                    provenientes de otros lenguajes orientados a objetos.
          \end{itemize}
    \item Gestión de Módulos:
          \begin{itemize}
              \item Antes de ES6, JavaScript no tenía un sistema nativo de módulos, lo que dificultaba la
                    organización y reutilización del código.
              \item La gestión de dependencias y la modularización del código eran complejas y dependían de
                    herramientas y patrones externos.
          \end{itemize}
    \item Problemas de Asincronía:
          \begin{itemize}
              \item El manejo de operaciones asíncronas en JavaScript, tradicionalmente mediante callbacks, puede
                    llevar al llamado "callback hell", haciendo que el código sea difícil de leer y mantener.
          \end{itemize}
\end{enumerate}

\subsection*{ El Surgimiento de TypeScript}

Para abordar muchos de estos problemas, Microsoft introdujo TypeScript en 2012. TypeScript es un superconjunto de
JavaScript que agrega tipos estáticos y otras características avanzadas, con el objetivo de mejorar la productividad
del desarrollo y la mantenibilidad del código.

\begin{enumerate}
    \item Tipos Estáticos:
          \begin{itemize}
              \item TypeScript introduce un sistema de tipos estáticos que permite a los desarrolladores definir y
                    verificar los tipos de variables, funciones y objetos en tiempo de compilación.
              \item Esto ayuda a identificar errores de tipo antes de que el código se ejecute, mejorando la
                    confiabilidad y reduciendo el número de errores en tiempo de ejecución.
          \end{itemize}
    \item Compatibilidad con JavaScript:
          \begin{itemize}
              \item TypeScript es un superconjunto estricto de JavaScript, lo que significa que cualquier código
                    JavaScript válido es también un código TypeScript válido.
              \item Esto permite a los desarrolladores adoptar TypeScript de manera incremental, comenzando con
                    archivos JavaScript existentes y agregando gradualmente tipos y características de TypeScript.
          \end{itemize}

    \item Mejor Herramientas y Soporte para IDE:
          \begin{itemize}
              \item TypeScript ofrece una integración superior con editores de código y entornos de desarrollo
                    integrados (IDE), proporcionando autocompletado, refactorización y navegación mejorados.
              \item Las herramientas de desarrollo se benefician del conocimiento del sistema de tipos, lo que
                    facilita la escritura y mantenimiento del código.
          \end{itemize}

    \item Clases y Herencia:
          \begin{itemize}
              \item TypeScript introduce una sintaxis de clases más familiar para los desarrolladores que provienen
                    de otros lenguajes orientados a objetos, mejorando la legibilidad y el uso de patrones de diseño
                    orientados a objetos.
              \item  Soporta herencia, encapsulación y polimorfismo de manera más intuitiva y estructurada.
          \end{itemize}
    \item Módulos y ES6+:
          \begin{itemize}
              \item TypeScript soporta módulos ES6, permitiendo una mejor organización del código y gestión de
                    dependencias.
              \item Aprovecha las características modernas de JavaScript, asegurando que el código esté alineado con
                    las mejores prácticas actuales.
          \end{itemize}
    \item Asincronía Mejorada:
          \begin{itemize}
              \item TypeScript mejora el manejo de operaciones asíncronas mediante el uso de async/await,
                    proporcionando una sintaxis más clara y directa para el manejo de promesas y funciones asíncronas.
          \end{itemize}
\end{enumerate}

TypeScript ha surgido como una poderosa herramienta para superar muchas de las limitaciones estructurales de 
JavaScript. Al agregar tipos estáticos, mejorar la organización del código y proporcionar una mejor experiencia 
de desarrollo, TypeScript ayuda a los desarrolladores a escribir código más robusto, mantenible y escalable. Esto 
ha llevado a su adopción creciente en la industria, especialmente para proyectos grandes y complejos donde las 
ventajas de un sistema de tipos sólido son más evidentes.

\end{document}